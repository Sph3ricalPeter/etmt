\documentclass[a4paper]{article}

\usepackage[english]{babel}
\usepackage[utf8]{inputenc}
\usepackage{mathtools}
\usepackage{graphicx}
\usepackage{parskip}

\graphicspath{ {/mnt/tex/images/} }

\everymath{\displaystyle}

\title{Zápočtový test ZDM č.1}

\author{jméno: \texttt{\detokenize{______________}}}

\date{12.12.2021}

\begin{document}
\maketitle

\renewcommand{\abstractname}{Pokyny k vypracování}
\begin{abstract}
\noindent Test vypracovávnejte samostatně\\
Otázky mají následující typy: slovní odpověď, zaškrtávací s jednou správnou odpovědí, zaškrtávací s více správnými odpověďmi.\\
\end{abstract}

\begin{enumerate}
  \item O množině D platí (více správných odpovědí) \hfill (3b)\\
    \begin{enumerate}
      \item \texttt{[ ]} Je to nějaká blbost\\
      \item \texttt{[ ]} Množina je\\
      \item \texttt{[ ]} Nevím\\
    \end{enumerate}
\end{enumerate}


\begin{enumerate}
  \item Dnes je (jedna správná odpověď) \hfill (1b)\\
    \begin{enumerate}
      \item \texttt{[ ]} Pondělí\\
      \item \texttt{[ ]} Není\\
      \item \texttt{[ ]} Neděle\\
      \item \texttt{[ ]} Já vážně nevím, teď jsem vstal ale řeknu ti $f(x)=2x+6$\\
    \end{enumerate}
\end{enumerate}

\begin{enumerate}
  \item Dnes je (slovní odpověď) \hfill (1b)\\


  
\end{enumerate}

\includegraphics{obr1}

\end{document}
